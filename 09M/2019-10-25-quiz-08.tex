\documentclass{ccg-topic}

\topic{Partner Quiz 09 Confidence Intervals}

\institution{University of Colorado}
\coursenum{2510-001}
\coursename{Introduction to Statistics}
\semester{Fall 2019}
\author{Colton Grainger}
\date{\today}
\email{colton.grainger@colorado.edu}
\thanks{These questions are from Joseph Timmer, Fall 2018.}

\begin{document}

\maketitle

\begin{itemize}
    \item Your name (print clearly in capital letters): \underline{\hspace{8cm}}
    \item This is a timed 15 minute partner quiz.  Please find a partner. 
    \item You will both receive the same grade.
    \item Thanks! Your partner's name (print clearly in capital letters): \underline{\hspace{6cm}}
    \item Note: You both will turn individual copies of this page.
    \item Please collaborate to write down $4$ (total!) arguments to respond to the following questions. 
    \item Each of your arguments will be graded out of 5 points ($4$ points for correctness, $1$ point for ``style''). 
    \item Y'all may split the writing any way y'all like. I will grade both pages at once. 
    \item Y'all do not need to duplicate each other's arguments. I only need $4$ arguments (total!) from the two of y'all (e.g., $2$ from one person and $2$ from another).
\end{itemize}



\section*{Graded Questions}

\begin{enumerate}
    \item If the standard deviation of the population increases, what happens to the margin for error of a confidence interval based on a random sample of the population? Explain why or give an explicit example.
    
    \vfill
    
    \item If the sample size increases, what happens to the margin for error of a confidence interval based on the sample? Explain why or give an explicit example.
    
    \vfill

    \newpage    

    \item If the level of confidence increases, what happens to the margin for error of a confidence interval based on a random sample? Explain why or give an explicit example.
    
    \vfill
    
    \item You may notice we do not ask a ``necessary sample size'' question for $t$ intervals. Why is this? (It might help to remember the how to compute a $t$-interval from, say, $n=25$ a sample size, $\overline{X} = 100$ a sample mean, and $S_{\overline{X}} = 10$ the sample standard deviation. \emph{Which of Student's $t$-distributions did you need to use?})

    \vfill    
\end{enumerate}

\end{document}

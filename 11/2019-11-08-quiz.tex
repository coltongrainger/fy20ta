\documentclass{ccg-topic}

\topic{Quiz Week 11}

\institution{University of Colorado}
\coursenum{2510-001}
\coursename{Introduction to Statistics}
\semester{Fall 2019}
\author{Colton Grainger}
\date{2019-11-08}
\email{colton.grainger@colorado.edu}

\newcommand{\answer}[1]{} 

\begin{document}

\maketitle

Your name (print clearly in capital letters): \underline{\hspace{8cm}}

\section*{Read Me Carefully}
\begin{itemize}
    \item This quiz is out of 10 points. 
    \item Please \textbf{answer each graded question}, return this quiz to me by \textbf{Friday, November 8th at 8:20am}.
    \item You may complete any of the \emph{additional questions} to earn back points on either this quiz or the Week 10 quiz.
    \item You should use your notes, a calculator, and a piece of scratch paper. You do not need to show work.
    \item You may not collaborate. 
\end{itemize}

\section*{Graded Questions}

\begin{enumerate}

\item (2 points) It is estimated that 2\% of the world's population has green eyes. How many CU students would be needed to estimate the percentage of CU students with green eyes within 0.1\% at 95\% confidence? 

\framebox[15cm][l]{Your answer: }
\vfill

\item (1 point) Is it feasible to obtain a sample of this size from CU students? 

\framebox[15cm][l]{Your answer: }
\vfill

\item Suppose you are going to be testing a hypothesis percentage of people with green eyes and the null hypothesis is that $p=0.02$. If your random sample of size $n=200$ found 6 people with green eyes, is this an appropriate sample to use for a hypothesis test? 

\begin{enumerate}
\item (1 point) Yes or no? 

\framebox[15cm][l]{Your answer: }
\vfill

\item (1 point) Explain.

\framebox[15cm][l]{Your answer: \\\ }
\vfill
\end{enumerate}

\item If your random sample of size $n=300$ found 6 people with green eyes, is this an appropriate sample to use for a hypothesis test?

\begin{enumerate}
\item (0 points) Yes or no? 

\framebox[15cm][l]{Your answer: Yes.}
\vfill

\item (1 point) Explain.

\framebox[15cm][l]{Your answer: }
\vfill
\end{enumerate}

\item (2 points) If the alternate hypothesis is that more than 2\% of the population has green eyes, what conclusion would you make at the 5\% significance level based on the second sample (from question 4)?

\framebox[15cm][l]{Your answer: }
\vfill

\item (2 points) If you perform 20 independent hypotheses tests, each at the 5\% level of significance,%
    \footnote{%
    Hint: What does the level of significance $\alpha$ mean in hypothesis testing? What is ``Type I error''? 
    }
what is the probability at least one of them rejects the null hypothesis?

\framebox[15cm][l]{Your answer: }
\vfill
\end{enumerate}

\section*{Additional Questions}



(6 points possible) What are the mathematical expression for the respective margin of errors \[E = z^\star \times SE\] or \[E = t^\star \times SE\] in the third column of the following table? 
\begin{center}

\begin{tabular}{| p{5cm} | p{2cm} | p{8cm} |}
\hline
\textbf{With a $c$-level of confidence, I want to estimate...} & \textbf{calculator function...} & \textbf{In this case, the margin of error $E$ is...} \\
\hline
\hline
A population mean $\mu$ with known population standard deviation $\sigma$ & {\answer \texttt{ZInterval}} &  \\[1.5cm]
\hline
A population mean $\mu$ with unknown population standard deviation &{\answer \texttt{TInterval}} & \\[1.5cm]
\hline
A population proportion $p$ & {\answer \texttt{1-PropZInt}}& \\[1.5cm]
\hline
A difference of population means $\mu_1 - \mu_2$ with unknown population standard deviation $\sigma$  &{\answer \texttt{2-SampZInt}} &\\[1.5cm]
\hline
A difference of population means $\mu_1 - \mu_2$ with known population standard deviation $\sigma$  & {\answer \texttt{2-SampTInt}}&\\[1.5cm]
\hline
A difference of population proportions $p_1 - p_2$ &{\answer \texttt{2-PropZInt}} &\\[1.5cm]
\hline
\end{tabular}

\end{center}

\end{document}

\documentclass{article}
\usepackage{graphicx, color}


\setlength{\textwidth}{6.5in}
\setlength{\textheight}{8.0in}
\setlength{\oddsidemargin}{0in}
\setlength{\evensidemargin}{0in}
\setlength{\parskip}{2ex}
\setlength{\parindent}{0in}

%To display answers, replace "white" with "red" here;
\newcommand{\answer}[1]{\color{white}#1}

\begin{document}
\pagestyle{myheadings}\markright{
CU Boulder \hspace{0.5in} MATH 2510 - Introduction to Statistics }

\begin{center}
\textbf{\underbar{Worksheet 16 Exit Sheet}}
\end{center}

Moderator:{\underbar{\hspace{2in}}} \hfill  Recorder:{\underbar{\hspace{2in}} 

\bigskip

Reporter:{\underbar{\hspace{2in}}

As a group, come up with answers to the following questions.

\begin{enumerate}

    \item It is estimated that 2\% of the world's population has green eyes. How many CU students would be needed to estimate the percentage of CU students with green eyes within 0.1\% at 95\% confidence? 
    
    Does this sound feasible?
    
    \vfill
    
    \item Suppose a research report is to include 20 confidence intervals, each at the 95\% level. Although one would expect some relationships, suppose the intervals are based on independent statistics.
    
        \begin{enumerate}
        
            \item About how many of the intervals would you expect actually to include the true value of the parameter being estimated?
            
            \vfill
            
            \item What is the probability that all 20 intervals will contain the true values of the parameters being estimated?
            
            \vfill
        
        \end{enumerate}
        
        
\end{enumerate}
	
\end{document}
\documentclass{article}
\usepackage{graphicx, color}


\setlength{\textwidth}{6.5in}
\setlength{\textheight}{8.0in}
\setlength{\oddsidemargin}{0in}
\setlength{\evensidemargin}{0in}
\setlength{\parskip}{2ex}
\setlength{\parindent}{0in}

%To display answers, replace "white" with "red" here;
\newcommand{\answer}[1]{\color{white}#1}

\begin{document}
\pagestyle{myheadings}\markright{
CU Boulder \hspace{0.5in} MATH 2510 - Introduction to Statistics }

\begin{center}
\textbf{\underbar{Worksheet 24 Exit Sheet}}
\end{center}

Moderator:{\underbar{\hspace{2in}}} \hfill  Recorder:{\underbar{\hspace{2in}} 

\bigskip

Reporter:{\underbar{\hspace{2in}}

As a group, come up with answers to the following questions.


Consider two samples with the following information:

\begin{center}
\textbf{Sample A}: $n=6$ and $r=0.90$ \hspace{5em} \textbf{Sample B}: $n=10$ and $r=0.90$
\end{center}

\begin{enumerate}
\item Compute the test statistic $t=\frac{r\sqrt{n-2}}{\sqrt{1-r^2}}$ for each sample.

\vfill

\item Compute the critical $t$ value for a 2-tailed test. 

\textit{Hint: Use the custom \texttt{TCRIT} prgm OR use \texttt{invT}. This is similar to looking up a critical value for a confidence interval.}

\vfill

\item Does either sample have a significant linear correlation at the 1\% significance level? Note that in both samples the correlation coefficient is the same.

\textit{Hint: If the test statistic is larger than the critical t value, the $P$-value of the test is less than the level of significance}

\vfill

\end{enumerate}

\end{document}
\documentclass{article}
\usepackage[utf8]{inputenc}

\begin{document}

\section*{2.1 Frequency Distributions, Histograms, Etc.}

\begin{enumerate}

    \item Describe how frequency tables organize data into classes and then lists the numbers of data points in each class
    
    \item Describe the procedure to finding class limits/widths for {\bf integer} valued data
    
        \begin{enumerate}
        
            \item $\displaystyle{class\;width = \frac{largest\;data\;value - smallest\;data\;value}{number of class limits}}$
            
            \item Then {\bf increase} this to the next integer, even if the above computation yields an integer. For example, $=4.4\rightarrow 5$ and $=7 \rightarrow 8$.
            
        \end{enumerate}
        
    \item How to compute limits,boundaries/midpoints of classes.
    
    \item Do an example of this, probably with a dozen data points and 3 classes.
    
    \item If the data is decimals, then multiply by an appropriate power of 10 to convert all data points to integers. Proceed as above. Then divide class limits, boundaries and midpoints by said power of 10.
    
    \item Histograms
    
    \item Relative frequency in tables and histograms
    
    \item Shapes of distributions
    
        \begin{itemize}
        
            \item Left-skewed data means the long-tail is on the {\bf left side} of the histogram.
            
            \item Right-skewed data means the long-tail is on the {\bf right side} of the histrogram.
            
        \end{itemize}
    
    \item Cumulative frequency tables
    
    \item Ogive (pronounced oh-jive and rhymes with hive)
    
\end{enumerate}

\section*{2.2 Types of Graphs}

\begin{enumerate}

    \item Bar Graphs
        \begin{enumerate}
            \item Bars represent frequencies of classes
            \item Bars are evenly spaced, even width.
        \end{enumerate}
    
    \item Circle Graphs (commonly known as Pie Charts)
    
        \begin{itemize}
        
            \item Only used for parts of a whole
            
        \end{itemize}
        
    \item Time series data/graphs
    
    \item Pareto graphs
    
        \begin{itemize}
        
            \item A special type of bar graph where height represents frequency of an event.
            
            \item The rectangles are arranged from most frequent to least frequent.
            
            \item Not necessarily left-to-right; can be arranged vertically.
            
        \end{itemize}
        
\end{enumerate}

\section*{2.3 Stem and Leaf Displays}

\begin{enumerate}

    \item Pay attention to the {\bf key}.
    
    \item Note the leaves need not be ordered
    
    \item Can also be back-to-back to show data tables for two different groups.
    
\end{enumerate}

\end{document}

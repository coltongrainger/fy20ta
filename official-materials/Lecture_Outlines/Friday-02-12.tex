\documentclass{article}

\usepackage[super]{nth}
\usepackage{amsmath, amssymb}

\setlength{\textwidth}{6.5in}
\setlength{\textheight}{8.0in}
\setlength{\oddsidemargin}{0in}
\setlength{\evensidemargin}{0in}
\setlength{\parskip}{2ex}
\setlength{\parindent}{0in}

\begin{document}

The primary goal here is to give the students an outline of material to review for the Midterm on Wednesday. Since they have a quiz this day, I will not have enough time to fill in details. I plan to write this out and if someone has a specific question on something, I will then go into detail.

\begin{enumerate}

    \item Remind them that Chapter Review (Chapter 4) on WebAssign is due tonight AND although Chapter Review (Chapter 5) is not due until 2/19, the midterm on 2/17 covers Chapter 5 material, so it will be to their advantage to complete it before the midterm.

    \item As far as resources during the midterm:
        
        \begin{itemize}
        
            \item Calculator (with NO internet access) When a calculator function is used, \textbf{work required is to write the function and input values used.} I do not supply calculators. If they show up without one on exam day, then they will suffer. Phones will not be allowed as a substitute.
            
            \item Formula sheet: A copy of the ``Frequently Used Formulas" for Chapter 1-5 (as shown in the back cover of the textbook; they can even see that in the online textbook).
            
        \end{itemize}
        
    \item Resources for more practice problems:
        
        \begin{itemize}
        
            \item WebAssign problems with ``Practice Another Version".  Not all problems are programmed with this option in WebAssign.
            
            \item Odd-numbered problems from the textbook where the correct answer can be checked in the back of the textbook (even the online version).
            
        \end{itemize}
    
    \item The worksheet on Monday will be a ``mini-midterm" useful for review. Homework for Monday is to start studying for the midterm.
    
\end{enumerate}


\section*{Outline of content by Chapter}

\begin{enumerate}

    \item CHAPTER 1
    
        \begin{enumerate}
        
            \item Vocabulary terms: Individual, population, quantitative variable, qualitative variable, statistic, parameter, descriptive statistics, inferential statistics
            
            \item Levels of measurement: Nominal, ordinal, interval, ratio
            
            \item Sampling techniques: Random, stratified, systematic, cluster, convenience, multi-stage
            
            \item Basics of experimental design: Observational study vs experiment, Control group, placebo and placebo effect
            
        \end{enumerate}
        
    \item CHAPTER 2
    
        \begin{enumerate}
        
            \item Displaying data: Frequency tables 
            
            \item Class limits, class boundaries, class width, midpoint, relative frequency, cumulative frequency
            
            \item Histograms \& ogives
            
            \item Symmetry and skewness of histogram
            
            \item Graphs: Bar graph, Pareto chart, Circle (pie) graph, Time-Series graph
            
            \item Stem-and-Leaf displays
            
        \end{enumerate}
        
    \item CHAPTER 3 : \texttt{1-Var Stats}
    
        \begin{enumerate}
        
            \item Central tendencies: Mean, median, mode, trimmed mean, weighted mean
            
            \item Variation
            
                \begin{itemize}
                    
                    \item Range, variance, standard deviation (statistic and parameter)
                    
                    \item Coefficient of variation
                    
                    \item Chebyshev’s Theorem
                
                \end{itemize}
                
            \item Percentiles and Box-and-Whisker
            
            \item Quartiles, IQR, 5-number summary
            
        \end{enumerate}
        
    \item CHAPTER 4
    
        \begin{enumerate}
        
            \item Elementary probability theory
            
                \begin{itemize}
                
                    \item Sample space, notation $P(A)$, $P(A^c)$, equally likely outcomes, using relative frequency
                    
                    \item $P(A \mbox{ or } B)$, $P(A \mbox{ and } B)$, $P(A|B)$, independence, mutual exclusivity
                    
                \end{itemize}
                
        \end{enumerate}
        
    \item CHAPTER 5 
    
        \begin{enumerate}
        
            \item Discrete probability distribution
            
            \item Valid probability distribution, mean (expected value), standard deviation
            
            \item Binomial probabilities
            
                \begin{itemize}
                
                    \item Criteria of a binomial experiment
                    
                    \item Probability of exactly r successes, at most r successes, at least r successes, fewer than r successes, more than r successes, etc.
                    
                \end{itemize}
        \end{enumerate}
        
\end{enumerate}

\end{document}
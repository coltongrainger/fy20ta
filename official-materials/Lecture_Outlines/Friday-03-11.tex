\documentclass{article}

\usepackage[super]{nth}
\usepackage{amsmath, amssymb}
\usepackage{hyperref}

\setlength{\textwidth}{6.5in}
\setlength{\textheight}{8in}
\setlength{\oddsidemargin}{0in}
\setlength{\evensidemargin}{0in}

\setlength{\parskip}{2ex}
\setlength{\parindent}{0in}

\begin{document}

\section*{Internet Visual for Confidence Intervals}

\begin{enumerate}

    \item Start with an overview of what this whole confidence interval and confidence level thing is all about. This applet could be effective in confirming what idea of the c-level, also as a way to compare how the size of the interval changes with changes to the variables.
    
    \item The website is \url{http://www.rossmanchance.com/applets/ConfSim.html}
    
    \item Start with Means, Normal, z with $\sigma$.
    
        \begin{itemize}
        
            \item Show 1 sample
            
            \item Show 10 samples
            
            \item Show 100 samples, maybe a few times to show how the number of ``good" confidence intervals can vary.
            
            \item Change Conf level
            
            \item Change $n$
            
        \end{itemize}
        
    \item Reset and look at Means, Normal, $t$. Note how they can have different lengths. (this is sometimes subtle)  Why?
    
\end{enumerate}

\section*{Section 7.3 Confidence Interval for a Proportion}

\begin{enumerate}

    \item This time, rather than a mean, we are estimating a population proportion, like ``What percentage of all college students change their major at least once in their first four years?"  which is different than ``What is the average number of times a college student changes their major within their first four years?"
    
    \item The calculator function is \texttt{1-PropZInt}.
    
    \item In the reading watch out for
    
        \begin{itemize}
        
            \item The requirement on the sample size, it is more complicated than just $n \geq 30$.
            
            \item The formula for E.
            
            \item The formulas for finding sample size.
            
            \item Interpreting poll results.
            
        \end{itemize}
        
\end{enumerate}

\newpage

\section*{Section 7.4 Confidence Intervals for Differences}

\begin{enumerate}

    \item As our last look at confidence intervals, we look at differences (between two means or between two proportions), as a way to tell if two populations are different.
    
    \item This section refers only to independent samples, but check out the reading for the definitions as it will matter later (when we do tests).
    
    \item \texttt{2-SampZInt}, \texttt{2-SampTInt}, and \texttt{2-PropZInt} are the calculator functions.
    
    \item In the reading watch out for
    
        \begin{itemize}
        
            \item The degrees of freedom for Tint
            
            \item The criteria on sample size (it is again different for $p$ than for $\mu$).
            
            \item The interpretation of the confidence interval
                
            \begin{itemize}
                
                \item When the interval contains only negative values
                
                \item When the interval contains only positive values
                
                \item When the interval contains both positive and negative values.
                
            \end{itemize}
            
        \end{itemize}
        
\end{enumerate}

\end{document}
\documentclass{article}

\usepackage[super]{nth}
\usepackage{amsmath, amssymb}
\usepackage{hyperref}

\setlength{\textwidth}{6.5in}
\setlength{\textheight}{8in}
\setlength{\oddsidemargin}{0in}
\setlength{\evensidemargin}{0in}

\setlength{\parskip}{2ex}
\setlength{\parindent}{0in}

\begin{document}

\section*{7.1 Estimating $\mu$ when $\sigma$ is known}

\begin{enumerate}

    \item The first look at confidence intervals assumes that we know $\sigma$.  (This might seem a bit contrived, because why would we know $\sigma$ if we don’t know $\mu$.)
    
    \item The TI \texttt{Z-INTERVAL} function will compute the confidence interval, either by entering the actual list of sample data or by entering the sample mean. For that reason some of the details of the formulation developed in the reading are not so critical, but understanding what the function does and how to interpret the result is critical.
    
    \item When reading, pay special attention to
    
        \begin{itemize}
        
            \item The formula for E in this case, as it should make sense why this is the correct formula.
            
            \item The definition of $z_c$, as this should also be something that you already know how to compute.
            
            \item The interpretation of a confidence interval, as it is easy to misinterpret what the interval found means.
            
        \end{itemize}
        
\end{enumerate}

\section*{7.2 Estimating $\mu$ when $\sigma$ is unknown}

\begin{enumerate}

    \item In this case, a seemingly more common case, we don’t know $\sigma$. So, we must estimate the value of $\sigma$ too. This translates into a slightly larger margin of error to compensate for the potential error in our guess of $\sigma$.
    
    \item Rather than the standard normal distribution, we use the Student's $t$-distributions. There is a slightly different distribution for each sample size, but they are all bell-shaped.  As $n$ gets larger, the $t$-distributions approach the standard normal distribution.
    
    \item The formulation and interpretation of a confidence interval in this case is very similar to that when $\sigma$ is known, just the distributions from which the critical values are determined has changed.
    
    \item The \texttt{T-INTERVAL} function the TI can do the work for us.
    
    \item When reading, pay special attention to
    
        \begin{itemize}
        
            \item The formula for E (which should look VERY similar to that from Section 7.1).
            
            \item The formula for degrees of freedom.
            
        \end{itemize}
        
    \item If your calculator does not already have an \texttt{InvT} function, there is a screencast available about how to create a program in your calculator that does the job.
    
\end{enumerate}

\end{document}
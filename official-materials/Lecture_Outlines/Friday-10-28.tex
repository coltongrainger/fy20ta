\documentclass{article}

\usepackage[super]{nth}
\usepackage{amsmath, amssymb}
\usepackage{hyperref}

\setlength{\textwidth}{6.5in}
\setlength{\textheight}{8in}
\setlength{\oddsidemargin}{0in}
\setlength{\evensidemargin}{0in}

\setlength{\parskip}{2ex}
\setlength{\parindent}{0in}

\begin{document}

\section*{Hypothesis Testing}

In Chapter 7, we estimated the value of population parameters (mean and proportion) using confidence intervals. Another method of statistical inference is to make decisions concerning the value of a population parameter, which we do in Chapter 8 with hypothesis testing.

\begin{enumerate}

  \item Suppose that you roll a regular six-sided die 600 times. About how many times would you expect to see a 4 rolled within those 600?
    
    \begin{enumerate}
    
      \item If you saw 105 rolls that were 4, would this be surprising\ldots enough to question the fairness of the die?

      \item What if you saw 595 rolls that were 4, would this be surprising\ldots enough to question the fairness of the die?

      \item Where would you draw the line between ``not so surprising" and ``surprising"?
      
    \end{enumerate}
    
  \item The basic idea in hypothesis testing is to start with an assumption of what ''should" happen and to draw a line on what extreme outcomes would be ``surprising". If the random sample indicates a ''surprising" result, we have evidence to abandon our assumption\ldots if the ransom sample indicates a ``not so surprising" result, we do not have adequate evidence to abandon our assumption and we must stick with it.
  
  \item Note, as with the case of the rolls of the die, the result of the random sample may be very ``surprising" (595 of our 600 rolls were 4), but it will never serve as PROOF that our assumption is wrong (as it's possible that this 595/600 happens with a completely fair die).
  
\end{enumerate}

\section*{8.1 Introduction to Statistical Tests}

\begin{enumerate}

  \item This section introduces the language and formalizes the concepts of ``drawing a line" and interpreting the results of the test.
  
  \item When reading, pays close attention to
  
    \begin{enumerate}
    
      \item The notation and definition/usage of the null hypothesis and the alternate hypothesis. 
      
      \item How we categorize the test as right-tailed, left-tailed, or two-tailed.
      
      \item What the P-value measures and how it is used to draw the conclusion of the test.
      
      \item The usage and meaning of the conventional language of ``Reject $H_0$" and ``Fail to Reject $H_0$".
      
    \end{enumerate}
    
  \item I plan to save the discussion of types of errors until Monday when they should have a slightly better sense of what this is all about.

\end{enumerate}

\newpage

\section*{8.2 Testing the Mean}

\begin{enumerate}

  \item The calculator functions that will be useful are \texttt{Z-Test} and \texttt{T-Test}. Like \texttt{Z-Interval} and \texttt{T-Interval}, one is used with $\sigma$ is known and the other when $\sigma$ is unknown.
  
  \item They may find the ``critical regions" method helpful in solidifying the concepts of hypothesis testing, but they will be required to compute and interpret P-values as well. So, this ``critical regions" method should be considered a secondary method.
  
\end{enumerate}

\end{document}
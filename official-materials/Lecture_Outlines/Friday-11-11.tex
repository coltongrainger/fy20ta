\documentclass{article}

\usepackage[super]{nth}
\usepackage{amsmath, amssymb}
\usepackage{hyperref}

\setlength{\textwidth}{6.5in}
\setlength{\textheight}{8in}
\setlength{\oddsidemargin}{0in}
\setlength{\evensidemargin}{0in}

\setlength{\parskip}{2ex}
\setlength{\parindent}{0in}

\begin{document}

The primary goal here is to give the students an outline of material to review for the Midterm on Wednesday. Since they have a quiz this day, I will not have enough time to fill in details. I plan to write this out and if someone has a specific question on something, I will then go into detail.

\begin{enumerate}

    \item Remind them that although Chapter Review (Chapter 8) is not due until 11/18, the midterm on 11/16 covers Chapter 8 material, so it will be to their advantage to complete it before the midterm.

    \item As far as resources during the midterm:
        
        \begin{itemize}
        
            \item Calculator (with NO internet access) When a calculator function is used, \textbf{work required is to write the function and input values used.} I do not supply calculators. If they show up without one on exam day, then they will suffer. Phones will not be allowed as a substitute.
            
            \item Formula sheet: A copy of the ``Frequently Used Formulas" for Chapter 1-8 (as shown in the back cover of the textbook; they can even see that in the online textbook).
            
            \item The ``Calculator Functions Syntax" sheet on D2L. This just states the necessary parameters for each function with no explanation of what they mean.
            
        \end{itemize}
        
    \item Resources for more practice problems:
        
        \begin{itemize}
        
            \item WebAssign problems with ``Practice Another Version".  Not all problems are programmed with this option in WebAssign.
            
            \item Odd-numbered problems from the textbook where the correct answer can be checked in the back of the textbook (even the online version).
            
        \end{itemize}
    
    \item The worksheet on Monday will be a ``mini-midterm" useful for review. Homework for Monday is to start studying for the midterm.
    
\end{enumerate}


\section*{Outline of content by Chapter}

\begin{enumerate}

    \item CHAPTER 6: Normal Curves and Sampling Distributions
    
        \begin{enumerate}
        
        \item Properties of Normal distribution
        
        \item Empirical Rule: Approximation of how much area is between $\mu \pm k\sigma$ for $k=1,2,3$.
        
        \item Control Charts, we have 3 ``warning signs".
        
            \begin{itemize}
            
                \item Any point beyond $\pm 3\sigma$.
                
                \item Nine consecutive points all above or all below $\mu$.
                
                \item Two of Three consecutive points beyond $\pm 2\sigma$.
                
            \end{itemize}
            
        \item Conversion to standard normal distribution: $z=\frac{x-\mu}{\sigma}$ or $x=\mu + z\sigma$. A $z$-score represents the number of standard deviations from normal an observation is.
        
        \item Using \texttt{normalcdf} to compute probabilities. Use $\pm 1 E 99$ for bounds $\pm\infty$.
        
        \item The Central Limit Theorem
        
        \end{enumerate}
        
    \newpage
    
    \item CHAPTER 7: Estimation (or Confidence Intervals)
    
        \begin{enumerate}
        
            \item General Philosophy of an $x$\% confidence interval: A \textbf{process} that produces an interval which contains the desired parameter $x$\% of the time.
            
            \item This \textbf{does not} mean given an $x$\% confidence interval, the chance the interval contains the parameter is $x$\%. This is like saying a 99\% accurate test (which means the text gets the \textbf{correct} diagnosis 99\% of the time) that says you have a disease means the probability of you \textbf{actually having the disease} is 99\%; which is a false, but common conclusion people make.
            
            \item The above point is a very common misconception and is often propagated. I myself am guilty of this. The good news however is that the confidence intervals we produce in this class actually have this additional property, so it is probably worth mentioning the fallacy, but not spending much time on it.
            
            \item CI types:
            
                \begin{itemize}
                
                    \item CI for $\mu$, $z$ or $t$-based.
                    
                    \item CI for $p$ a population proportion. Recall the requirements!
                    
                    \item CI for $\mu_1 - \mu_2$, $p_1-p_2$
                    
                    \item Mention the relevant functions.
                    
                \end{itemize}
                
        \end{enumerate}
        
    \item CHAPTER 8: Hypothesis Testing
    
        \begin{enumerate}
        
            \item Null Hypothesis: The assumption that a parameter is equal to some value.
            
            \item Alternate Hypothesis: The belief being ``tested", phrased that the parameter is different (in some fashion) from what is assumed.
            
            \item $p$-value and $\alpha$, the level of significance.
            
            \item Error types
            
            \item Test types -- completely determined by $H_1$.
            
            \item Reject $H_0$ when $p <= \alpha$. Phrasing a proper conclusion.
            
            \item Hypothesis test types:
            
                \begin{itemize}
                
                \item $z$ and $t$ tests
                
                \item Paired vs. Unpaired data. Be sure to state the null hypothesis is \textbf{always} $H_0: \mu = 0$ for such tests.
                
                \end{itemize}
                
        \end{enumerate}
        
\end{enumerate}

\end{document}
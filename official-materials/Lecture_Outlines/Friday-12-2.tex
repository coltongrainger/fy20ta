\documentclass{article}

\usepackage[super]{nth}
\usepackage{amsmath, amssymb}
\usepackage{hyperref}

\setlength{\textwidth}{6.5in}
\setlength{\textheight}{8in}
\setlength{\oddsidemargin}{0in}
\setlength{\evensidemargin}{0in}

\setlength{\parskip}{2ex}
\setlength{\parindent}{0in}

\begin{document}

\section*{10.2 Chi-Square: Goodness of Fit}

\begin{enumerate}

  \item Start with an overview of what ``goodness of fit" means. The null hypothesis is always that the population ``fits". The more deviant the data is from the expected distribution the worse the fit.
  
  \item Although this is a test that is supported by the TI, I like to spend some time analyzing the formula for chi-square and the distribution. It just so easy to see what it is measuring and why large test statistic values imply small P-value.
  
  \item Note that we are using a different distribution (than standard normal or Student’s t).
  
  \item Note, we have yet another measure of degrees of freedom.
  
  \item The key calculator function is $\chi^2$\texttt{GOF-Test}.
  
\end{enumerate}

\section*{10.5 ANOVA}

\begin{enumerate}

    \item My intent here is NOT to really get them to delve into the ``why" and ``how" the derivation of the sample F ratio, as I think that would require more time that we have. So, you can note that in the reading they will see a BUNCH of steps and intermediate computations, but they will not be responsible for understanding all those details for this class.
    
    \item Talk about the null and alternate hypotheses of the test.
    
    \item Talk about the sample F ratio is a measure of the variance BETWEEN populations versus the variance WITHIN each population. When there is significant variance BETWEEN, our F ratio will be big (we are removing the impact of the variance within each separate population and basically isolating the difference between populations).
    
    \item Note that we are using yet another distribution.
    
    \item Note that evidence supporting that there is a difference between the populations provides no direct evidence on which population might be the different one.
    
    \item The key calculator function is \texttt{ANOVA}.
    
\end{enumerate}

\end{document}
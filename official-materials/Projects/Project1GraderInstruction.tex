\documentclass{article}
\usepackage[utf8]{inputenc}


\begin{document}
\begin{center}
\textbf{
{\large How to use the Project 1 Grader File}	
 }
\end{center}
	
\begin{enumerate}
    
    \item In order to use the grader, you will need to link the student's spreadsheet to the grader file. Initially, the grader was linked to a file on my computer.
    
    \item Download from D2L an Excel file for each group. 
    
    \item When you open the grader for the first time, it will prompt you to Edit Links. Click on Edit Links and choose Change Source... Select the student file you wish to use.
    
    \item Grade the group's worksheet. Grade using \textbf{the student's Excel file} for an answer key. Here we should be looking if they are giving an accurate summary of what their work states, not if their work is actually correct.
    
    \item The Rubric Tab will give the \textbf{minimum} number of points you should assign in each part. For anything that is not full credit, you need to check out.
    
    \item On the KEY tab, you will see what their answers should be. Next to each answer box (colored yellow) is a blue box. The grader file will say ``Check" if the student answer is not identical to the Grader answer.
        
        \begin{itemize}
        
            \item For each check, see what the student did wrong. Sometimes a student's answer will be correct, but not identical to the grader output. Pay close attention to such possibilities.
    
            \item Also, carrying over errors can be detected and given partial credit if warranted. For example, if they messed up the mean and standard deviation of coin flips, their Chebyshev interval still might be correct for their values and deserving full credit. Compare cell formulas to try and determine this.
            
        \end{itemize}
        
    \item To check for ROBUSTNESS, copy the data from the ROBUST tab in the grader and paste it over the data section \textbf{in the student's worksheet.} Begin the grading process again from step 5, using the ROBUST tab as the answer key.
    
        \begin{itemize}
        
            \item Note the graphs should update also.
            
            \item Naturally, if they had something wrong before you changed the data set, the cell is likely to still be wrong. We are looking to see if the cells \textbf{update after the new data set} is applied, even if still wrong.
            
        \end{itemize}
        
    \item Enter a grade for the group into D2L and move on to the next group. You can just Edit links in the grader file for each new spreadsheet.

    \item If you are unsure about anything, let me know and I can show you in person how it all works.
    
\end{enumerate}

\end{document}

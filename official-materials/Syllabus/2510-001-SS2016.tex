
%###############################################################################

  \documentclass[11pt]{article}
  \usepackage{url, longtable}
  \usepackage{hyperref}
 % \addtolength{\topmargin}{-0.50in}
  \addtolength{\textheight}{2\baselineskip}
  \setlength{\oddsidemargin}{-20pt}
  \setlength{\evensidemargin}{0pt}
  \setlength{\textwidth}{6.5in}
  \setlength{\textheight}{11in}
  \setlength{\parindent}{0pt}
  \setlength{\leftmargini}{0pt}
% \renewcommand{\baselinestretch}{1.1}

  \pagestyle{myheadings}
  \markright{\sc MATH 2510 --- Spring 2016 }

%###############################################################################

\begin{document}

%###############################################################################

\thispagestyle{plain}

\addtolength{\topmargin}{-1.0in}
\centerline{
  \framebox{\parbox{4.0in}{
    \vspace{6pt}
      \begin{centering}
             \Large       \bf S~T~A~T~I~S~T~I~C~S             \\
             \normalsize  \bf (~M~A~T~H~~~~2~5~1~0~)  \\[1pt]
             \normalsize  \bf S~P~R~I~N~G~~~2~0~1~6               \\[1pt]
      \end{centering}
    \vspace{6pt}
  }}
}

\vspace{10pt}

%===============================================================================

\normalsize
\small

\begin{longtable}{lp{5.2in}}
%
\textbf{Instructor:}
      & {\bf Name:}         \hspace{8.00ex} Joseph Timmer, Ph.D.
     \\
      & {\bf Office:}       \hspace{8.25ex} MATH~237
     \\
      & {\bf Office Hours:} \hspace{.20ex} See OPEN OFFICE HOURS in D2L News
     \\
      & {\bf Email:}        \hspace{8.00ex} joseph.timmer@colorado.edu
     \\[6pt]
%
%\textbf{Course Webpage:}& Desire2Learn: \url{http://learn.colorado.edu}
%     \\[6pt]

\textbf{Class Meetings:}
     &    ECCR 1B51
     \\
     &    MWF 8:00\/AM -- 8:50\/AM 
   \\[6pt]

\textbf{Text:}
      & \textbf{\textsl{Understandable Statistics}\/, 10th Edition}
         by  Brase and Brase
         Brooks/Cole, 2012.
         ISBN 0-8400-4838-6.
         Chapters 1-9, and 10 (as time allows)
      \\
      & $\triangleright$~Topics to be included:
      \\
      & $\circ\;$ Introduction to experimental design - including simulations and sampling methods
      \\
      & $\circ\;$ Organizing data - including frequency distributions, histograms, stem-and-leaf, circle graphs, and time-series
      \\
      & $\circ\;$ Describing data - including central tendency, variation, and percentiles
      \\
      & $\circ\;$ Probability theory - including compound events, conditional probability, trees, counting techniques, and the binomial probability distribution
      \\
      & $\circ\;$ Normal curves and sampling distributions - including areas under a normal curve and the Central Limit Theorem
      \\
      & $\circ\;$ Confidence intervals - including the mean when sigma is known or unknown, proportions, and differences of means or proportions
      \\
      & $\circ\;$ Hypothesis testing - including the mean, a proportion, paired data, and differences of means or proportions
      \\
      & $\circ\;$ Correlation and regression - including the best fit line, and inferences for correlation and regression parameters
      \\
      & $\circ\;$ (As time allows) Chi-square and $F$ distributions - including tests for independence, goodness-of-fit, testing variance, and ANOVA
      \\[6pt]
%
%
\textbf{Grading:}
      & $\triangleright$~Your course grade will be computed from:
      \\
      & $\circ\;$
        \makebox[6cm][s]{
        {\bf In-class Participation}
        \dotfill}
        {\bf $5 \%$}
        \dotfill {\bf $5 \%$}
      \\      
      & $\circ\;$
        \makebox[6cm][s]{
        {\bf Reading Assignments}
        \dotfill}
        {\bf $10 \%$}
        \dotfill {\bf $10 \%$}
      \\
      & $\circ\;$
        \makebox[6cm][s]{
        {\bf Chapter Reviews}
        \dotfill}
        {\bf $10 \%$}
        \dotfill {\bf $10 \%$}
      \\            
      & $\circ\;$
        \makebox[6cm][s]{
        {\bf Projects}
        \dotfill}
        {\bf $10 \%$}
        \dotfill {\bf $10 \%$}
      \\
      & $\circ\;$
        \makebox[6cm][s]{
        {\bf Quizzes}
        \dotfill}
        {\bf $15 \%$}
        \dotfill {\bf $15 \%$}
      \\
      & $\circ\;$
        \makebox[6cm][s]{
        {\bf 2 Midterm Exams}
        \dotfill}
        {\bf $15\%$ each }
        \dotfill {\bf  $30 \%$}
      \\
      & $\circ\;$
        \makebox[6cm][s]{
        {\bf Final Exam}
        \dotfill}
        {\bf $20 \%$}
        \dotfill {\bf $20 \%$}
      \\[8pt]
%
%
\textbf{Calculators:}
	& A calculator is required for this course.  {\bf We recommend that you purchase a TI-84 Plus graphing calculator for this course.}  You will be permitted to use other calculators (provided they do not have access to the internet), however, they may not have the full library of functions to which we will refer.
	\\[8pt]

\textbf{WebAssign:}
	& We will be using WebAssign to administer online assessments for Reading Assignments and Chapter Reviews.  If you were registered or waitlisted for this course as of the Friday before classes began then you should already be enrolled in the WebAssign course.  You will need to purchase an access code for continued access throughout the semester.  Details on accessing WebAssign is posted in the D2L News feed.  Note that within the WebAssign course is a complete e-book version of the textbook along with video resources.
	\\[8pt]
	
%
\newpage
%
	
\textbf{Participation:}
	& Much of our in-class time will be spent engaged in active learning components.  These components may include worksheets, discussion, student presentations of solutions, etc.  Your attendance and contribution are critical for the success of these learning tools, and, as such, a part of your grade in this course will be based on your participation.  As long as you come to class prepared, on a regular basis, and make your best effort to complete the in-class activities and homework, then you should do well in this part of your grade.
	\\[8pt]
       

\textbf{\parbox[t][1.0in][t]{1.0in}{Reading Assignments:}}
     & Preparing for class meetings by reading the textbook and watching video supplements to expose yourself to the concepts and examples in the material to be explored in class is an important step in the format of this class.  Towards that end, you will be given specific reading assignments in WebAssign consisting of questions from the textbook due before class on most Mondays and Wednesdays throughout the semester.  {\bf Because these assignments are meant to compel you to prepare for class on a given day, no make-up Reading Assignments are allowed, but the lowest four will be dropped.}
       \\[8pt]

\textbf{\parbox[t][1.0in][t]{1.0in}{Chapter Reviews:}}
     & As we conclude the exploration of the material in a chapter of the textbook, a set of problems will be assigned in WebAssign.  These assignments will be due by 11pm on Friday about one week after the completion of the material in class.  If you fail to complete a Chapter Review assignment by the deadline, you will be able to request a 24-hour extension (for a maximum of 5 days after the due date) within WebAssign.  {\bf A 25\% late penalty will be assessed on any problems completed after the deadline.}
       \\[8pt]

\textbf{Projects:}       
       & In an attempt to expose you to tools used for statistical computations and analysis, you will be assigned two projects which require the application of a software package.  The first project will involve the use of Excel and the second will involve the use of the programming language R.  You will be required to work in coordination with other students in the class on these projects.  {\bf A late penalty of 25\% per weekday will be assessed for any project submission received after the announced due date.}
       \\[8pt]

\textbf{Quizzes:}
      &  Individual, in-class quizzes will be administered every Friday (during weeks for which there is not a Midterm scheduled). {\bf No make-up quizzes will be permitted, but the lowest three scores will be dropped.}
     \\[8pt]
%
\textbf{Exams:}
      & $\triangleright$~\textbf{Midterm Exams}.\/
        \newline
        The two exams will be given in class on the following days:
        \newline
        \textbf{1st}: \textsl{Wednesday, February~17},
        \newline
        \textbf{2nd}: \textsl{Wednesday, April~13}
     \\
     \\
     & $\triangleright$~\textbf{Final Exam}.\/
        \newline
        	{\underbar{Section 001}}:     \textsl{Tuesday, May~3, 7:30\/PM -- 10:00\/PM}
    \\
    \\
    & There are no makeup midterm exams for any reason. If you miss an exam your final exam grade will be substituted for your grade on the missed exam. In general, if your final exam  grade is better than one of your midterm grades  it will substitute for the low grade. 
    \\[20pt]

%
\textbf{\parbox[t][1.0in][t]{1.0in}{Important Dates:}}
     & \begin{tabular}{lr}
      	Martin Luther King Jr. Day & January 18 \\
     	Last day to drop without a ``W" & January 27 \\
      	Deadline for final exam change with 3 or more finals & March 18 \\
      	Last day to drop course in MyCUInfo (``W'' recorded) & March 18 \\
      	Spring Break & March 21-25 \\
      	Last day of classes & April 29 \\
       \end{tabular}
   \\

\end{longtable}

%-----------------------------------------------------------------------------

\newpage
\normalsize

\textbf{\normalsize STUDENTS WITH DISABILITIES}

\small
If you qualify for accommodations because of a disability, please
submit a letter from Disability Services in a timely manner so
that your needs may be addressed.  (For exam accommodations provide your letter at least one week prior to the exam.)  Disability Services determines
accommodations based on documented disabilities.  Contact Disability 
Services at 303-492-8671 or by e-mail at \url{dsinfo@colorado.edu}.
If you have a temporary medical condition or injury, see Temporary Injuries under Quick Links at
Disability Services website {\footnotesize \url{http://disabilityservices.colorado.edu/}} and discuss your needs with your professor.

%-----------------------------------------------------------------------------

\bigskip

\textbf{\normalsize RELIGIOUS OBLIGATIONS}



Campus policy regarding religious observances requires that faculty
make every effort to reasonably and fairly deal with all students
who, because of religious obligations, have conflicts with scheduled
exams, assignments or required attendance.  See full
details at

{\footnotesize \noindent \url{http://www.colorado.edu/policies/observance-religious-holidays-and-absences-classes-andor-exams} }

%-----------------------------------------------------------------------------

\bigskip

\textbf{\normalsize  CLASSROOM AND COURSE-RELATED BEHAVIOR}

Students and faculty each have responsibility for maintaining an
appropriate learning environment. Students who fail to adhere to
such behavioral standards may be subject to discipline. Faculty have
the professional responsibility to treat all students with
understanding, dignity and respect, to guide classroom discussion
and to set reasonable limits on the manner in which they and their
students express opinions.  Professional courtesy and sensitivity
are especially important with respect to individuals and topics
dealing with differences of race, culture, religion, politics,
sexual orientation, gender variance, and nationalities.  Class
rosters are provided to the instructor with the student's legal
name. I will gladly honor your request to address you by an
alternate name or gender pronoun. Please advise me of this
preference early in the semester so that I may make appropriate
changes to my records.  See polices at

{\footnotesize \url{http://www.colorado.edu/policies/student-classroom-and-course-related-behavior} }  

and at

\noindent
{\footnotesize\url{http://www.colorado.edu/osc/sites/default/files/attached-files/osc_handbook_2015-16.pdf}. }

%-----------------------------------------------------------------------------

\bigskip
\textbf{\normalsize  DISCRIMINATION AND HARASSMENT}

The University of Colorado Boulder (CU-Boulder) is committed to maintaining a positive learning, working, and living environment. CU-Boulder will not tolerate acts of discrimination or harassment based upon Protected Classes or related retaliation against or by any employee or student. For purposes of this CU-Boulder policy, "Protected Classes" refers to race, color, national origin, sex, pregnancy, age, disability, creed, religion, sexual orientation, gender identity, gender expression, veteran status, political affiliation or political philosophy. Individuals who believe they have been discriminated against should contact the Office of Institutional Equity and Compliance (OIEC) at 303-492-2127 or the Office of Student Conduct and Conflict Resolution (OSC) at 303-492-5550. Information about the OIEC, the above referenced policies, and the campus resources available to assist individuals regarding discrimination or harassment can be found at 

\noindent {\footnotesize \url{http://www.colorado.edu/institutionalequity/}}.  

The full policy on discrimination and harassment contains additional information and can be found at 

\noindent {\footnotesize \url{http://www.colorado.edu/policies/discrimination-and-harassment-policy-and-procedures}.}

%-----------------------------------------------------------------------------

\bigskip
\textbf{\normalsize UNIVERSITY'S HONOR CODE}

All students of the University of Colorado at Boulder are
responsible for knowing and adhering to the academic integrity
policy of this institution. Violations of this policy may include:
cheating, plagiarism, aid of academic dishonesty, fabrication,
lying, bribery, and threatening behavior.  All incidents of academic
misconduct shall be reported to the Honor Code Council
(honor@colorado.edu; 303-725-2273). Students who are found to be in
violation of the academic integrity policy will be subject to both
academic sanctions from the faculty member and non-academic
sanctions (including but not limited to university probation,
suspension, or expulsion). Other information on the Honor Code can
be found at 

\noindent {\footnotesize \url{http://www.colorado.edu/policies/student-honor-code-policy} } 

and at

\noindent {\footnotesize \url{http://honorcode.colorado.edu/}.}

%=============================================================================
\newcommand{\lect}[2]
{\parbox[t]{2.0in}{\textbf{#1:} \\ \hspace*{0.25in} \parbox[t]{2.5in}{\footnotesize{ #2}} \\ }}
%\newcommand{\lect}[3]
%{\parbox[t]{2.0in}{\textbf{#1:} \\ \hspace*{0.25in} \parbox[t]{2.5in}{#2 \\ \footnotesize{\em #3}} \\ }}

\newpage
\normalsize

\begin{tabular}{|l||l||l|}
\multicolumn{3}{c}{\textbf{Tentative Schedule}} \\
\hline
\parbox{2.0in}{\begin{center}\textbf{MONDAY} \\ \end{center}} &
\parbox{2.0in}{\begin{center}\textbf{WEDNESDAY} \\ \end{center}} &
\parbox{2.0in}{\begin{center}\textbf{FRIDAY} \\ \end{center}} \\

\hline
{\lect{January 11}{Welcome to MATH 2510}} & 
{\lect{January 13}{Due: Reading Ch 1}} &  
{\lect{January 15}{Quiz 1 (Ch1)}} \\

\hline
{\lect{January 18}{No Class}} & 
{\lect{January 20}{Due: Reading Ch 2}} & 
{\lect{January 22}{Quiz 2 (Ch 2) \\ Due: Chapter Review Ch 1}}  \\

\hline
{\lect{January 25}{Due: Reading Sec 3.1}} & 
{\lect{January 27}{Due: Reading Sec 3.2-3.3}} & 
{\lect{January 29}{Quiz 3 (Sec 3.1-3.3) \\ Due: Chapter Review Ch 2}}  \\

\hline
{\lect{February 1}{Due: Reading Sec 4.1}} & 
{\lect{February 3}{Due: Reading Sec 4.2}} & 
{\lect{February 5}{Quiz 4 (Sec 4.1-4.2) \\ Due: Chapter Review Ch 3}}  \\

\hline
{\lect{February 8}{Due: Reading Sec 5.1}} & 
{\lect{February 10}{Due: Reading Sec 5.2-5.3}} & 
{\lect{February 12}{Quiz 5 (Sec 5.1-5.3) \\ Due: Chapter Review Ch 4}}  \\

\hline
{\lect{February 15}{Due: Project 1}} & 
{\lect{February 17}{Midterm 1}} & 
{\lect{February 19}{Due: Chapter Review Ch 5}}  \\

\hline
{\lect{February 22}{Due: Reading Sec 6.1}} & 
{\lect{February 24}{Due: Reading Sec 6.2-6.3}} & 
{\lect{February 26}{Quiz 6 (Sec Sec 6.1-6.3)}}  \\

\hline
{\lect{February 29}{Due: Reading Sec 6.4}} & 
{\lect{March 2}{Due: Reading Sec 6.5}} & 
{\lect{March 4}{Quiz 7 (Sec 6.4-6.5)}}  \\

\hline
{\lect{March 7}{Due: Reading Sec 7.1}} & 
{\lect{March 9}{Due: Reading Sec 7.2}} & 
{\lect{March 11}{Quiz 8 (Sec 7.1-7.2) \\ Due: Chapter Review Ch 6}}  \\

\hline
{\lect{March 14}{Due: Reading Sec 7.3}} & 
{\lect{March 16}{Due: Reading Sec 7.4}} & 
{\lect{March 18}{Quiz 9 (Sec 7.3-7.4)}}  \\

\hline
{\lect{March 21}{No Class}} & 
{\lect{March 23}{No Class}} & 
{\lect{March 25}{No Class}}  \\

\hline
{\lect{March 28}{Due: Reading Sec 8.1}} & 
{\lect{March 30}{Due: Reading Sec 8.2}} & 
{\lect{April 1}{Quiz 10 (Sec 8.1-8.2) \\ Due: Chapter Review Ch 7}}  \\

\hline
{\lect{April 4}{Due: Reading Sec 8.3}} & 
{\lect{April 6}{Due: Reading Sec 8.4-8.5}} & 
{\lect{April 8}{Quiz 11 (Sec 8.3-8.5)}}  \\

\hline
{\lect{April 11}{Due: Project 2}} & 
{\lect{April 13}{Midterm 2}} & 
{\lect{April 15}{Due: Chapter Review Ch 8}}  \\

\hline
{\lect{April 18}{Due: Reading Sec 9.1-9.2}} & 
{\lect{April 20}{Due: Reading Sec 9.3}} & 
{\lect{April 22}{Quiz 12 (Sec 9.1-9.3)}}  \\

\hline
{\lect{April 25}{Due: Reading Sec 10.2}} & 
{\lect{April 27}{Due: Reading Sec 10.5}} & 
{\lect{April 29}{Quiz 13 (Sec 10.2, 10.5) \\ Due: Chapter Review Ch 9}}  \\

\hline
\end{tabular}

%###############################################################################

\end{document}

%###############################################################################
\documentclass{article}

\usepackage[super]{nth}
\usepackage{amsmath, amssymb}

\setlength{\textwidth}{6.5in}
\setlength{\textheight}{8in}
\setlength{\oddsidemargin}{0in}
\setlength{\evensidemargin}{0in}

\setlength{\parskip}{2ex}
\setlength{\parindent}{0in}

\begin{document}

\begin{enumerate}

\item The content of 6.4 and 6.5 are really the heart of what makes almost all the rest of the material in the class work. So, the big picture of what is happening is valuable. 

\item We are focusing on the normal distribution. Review the shape, mean and standard deviation of this distribution. Even though the population we inquire about may not be normally distributed, this distribution is none-the-less important.

\item If we have a random sample from a population, we can certainly compute the mean of that sample, but how well can we expect that sample mean to represent the population mean? 

\end{enumerate}

\section*{6.4 \& 6.5 Sampling Distributions and the Central Limit Theorem}

\begin{enumerate}

\item If we were able to take every possible random sample of a specified size $n$ from a population and compute the sample mean for each of those samples, what would that distribution of sample means look like?

\item The Central Limit Theorem tells us that we can expect that (with certain restrictions) the distribution is normal. This is a pretty profound and powerful result. No matter how scattered the population itself is, with some proper guidelines, we can expect that the sample means to be well behaved.

\item Because of this, we can determine the probability of finding a sample with a specific mean (assuming we know the population mean). This is very important as it is the cornerstone of why confidence intervals and hypothesis testing are sound reasoning.

\item The middle (mean) of that distribution of sample means is equal to the population mean itself. So, although there are some sample means that lie far off in the tails of the distribution, they aren't very likely.

\item Further, the standard deviation of that distribution of sample means is equal to the population standard deviation divided by the square root of the sample size. So this implies that the larger the sample size, the smaller the spread of the sample means, which should make sense. The larger the sample size, the more likely it represents the population well.

\end{enumerate}

\end{document}
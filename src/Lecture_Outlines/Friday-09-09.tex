\documentclass{article}

\usepackage[super]{nth}
\usepackage{amsmath, amssymb}
\usepackage{hyperref}

\setlength{\textwidth}{6.5in}
\setlength{\textheight}{8in}
\setlength{\oddsidemargin}{0in}
\setlength{\evensidemargin}{0in}

\setlength{\parskip}{2ex}
\setlength{\parindent}{0in}

\begin{document}

\section*{4.1 What is Probability?}

\begin{enumerate}

    \item Summary box on page 137 pretty much sums up the key points.

    \item A statistical experiment is any random activity that results in a definite outcome. Give examples like drawing cards, rolling dice, spinning the big wheel on ``The Price is Right", etc.
    
        \begin{itemize}
            
            \item Sample space, simple event, event. Give an example of these terms with dice rolling.
            
            \item Probability will always be a value between 0 and 1 (inclusive). State when an event has probability 0 or 1.
            
            \item I don’t care if students use fractions, decimals, or percentages.
            
        \end{itemize}
        
    \item Notation
    
        \begin{itemize}
            
            \item $P(A)$.
            
            \item $P(A^c)$.
            
            \item $P(A)+P(A^c)=1$.
            
        \end{itemize}
        
    \item Computed by
    
        \begin{itemize}
        
            \item Theoretical design and counting simple events.  Be sure to emphasize formula of $$P(A)=\frac{\textnormal{Number of outcomes in } A}{\textnormal{Number of all possible outcomes.}}.$$
            
            
            \item Empirical data and relative frequencies. Include the statement of the law of large numbers. Also emphasize that these methods will always \textbf{approximate} the probability of an event and may not ever get with total accuracy.
            
        \end{itemize}
        
    \item Why is probability relevant to statistics?
    
        \begin{itemize}
        
            \item Probability is about assigning a likelihood to a particular outcome of an unknown experiment. The most relevant situation is assigning probabilities to sampling from a \textbf{known} population.
            
            \item Statistics is about using the results of sampling to infer information about an \textbf{unknown} population.
            
        \end{itemize}
    
\end{enumerate}

\newpage

\section*{4.2 Some Probability Rules---Compound Events}

\begin{enumerate}

    \item There are some formulas in this section, but often I try to de-emphasize them in trade for ``careful counting".
    
    \item The concept of conditional probability is always tricky for students. Carefully explain the concept with emphasis that for $P(A | B)$, the event $B$ is \textbf{known} to have happened.
    
    \item Note the two key terms
    
        \begin{itemize}
        
            \item Mutually exclusive
            
            \item Independent
            
        \end{itemize}
        
    \item Be sure to explain the difference between to conjunctions \textbf{or} and \textbf{and}. The use of a Venn diagram will be of help.
    
    \item Describe a standard deck of 52 cards.
    
        \begin{itemize}
        
            \item Show how to compute $P(\mbox{Ace})$, $P(\mbox{Heart})$, \\ $P(\mbox{Ace or Heart}),$ $P(\mbox{Ace and Heart})$, $P(\mbox{Ace} | \mbox{Heart})$.
            
        \end{itemize}
        
    \item Provide a contingency table and compute some probabilities with it. 
    
    \renewcommand{\arraystretch}{1.2}
    
	\begin{center}
	
	\begin{tabular}{ccccc}
	& \multicolumn{3}{c}{\textbf{Political Affiliation}} \\ \cline{2-4}
	{\textbf{ Employee Type}} & Democrat (D) & Republican (R) & Independent (I) & Row Total\\	\hline
	Executive (E) \hfill& 5 & 34 & 9 & 48 \\
	
	\begin{tabular}{@{}r@{}} Production \\ Worker \end{tabular} (PW) \hfill & 63 & 21 & 8 & 92 \\
	\hline
	Column Total \hfill  & 68 & 55 & 17 & 140 \\
	\end{tabular}
	\end{center}
	
	    \begin{itemize}
	    
	        \item $P(D)$ and $P(E)$.
	        
	        \item $P(D \mbox{ and } E)$.
	        
	        \item $P(D | E)$.
	        
	   \end{itemize}
	   
\end{enumerate}
\end{document}
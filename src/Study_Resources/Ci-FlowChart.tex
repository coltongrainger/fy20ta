\documentclass[letterpaper,12pt,landscape]{article}
\usepackage[utf8x]{inputenc}
\usepackage{amsmath}
\usepackage[letterpaper,margin=1in,includehead=true]{geometry}
\usepackage{comment}
\usepackage{graphicx}
\usepackage{fancyhdr}
\usepackage{enumerate}
\pagestyle{fancy}
\usepackage{color}
\usepackage{wrapfig}
%\usepackage{enumitem}
\usepackage{tikz, tikz-3dplot, pgfplots}
\usepackage{tkz-graph}
\usetikzlibrary{calc,matrix}
\pgfplotsset{compat=newest}
%\usepackage{marvosym}
%\usepackage{titling}
%\setlength{\droptitle}{-1.5in}

%authored by Noah Nelson Williams 2015

\newcommand{\cloud}[1]{%
\begin{tikzpicture}[very thick, scale = #1]
	\draw[fill=gray!20] (1,0) circle (.25);	
	\draw[fill=gray!20] (.7,.3) circle (.25);	
	\draw[fill=gray!20] (.5,.2) circle (.25);	
	\draw[fill=gray!20] (.6,-.1) circle (.25);	
	\draw[fill=gray!20] (.2,0) circle (.25);	
	\draw [fill=gray,color=gray!20](.6,0) circle (.25);	
	\draw [fill=gray,color=gray!20](.5,.1) circle (.25);	
	\draw [fill=gray,color=gray!20](.65,.2) circle (.25);	
\end{tikzpicture}%
}


\newcommand\train[2]{%
\ifnum#2>0%
%\caboose{#1}%
\phantom{\traincar{#1/10}}%
\foreach \index in {1,...,#2}{%
\traincar{#1}%
\phantom{\traincar{#1/10}}%
}%
\fi%
\begin{tikzpicture}[scale=#1,every node/.style={scale=#1}]
	\draw[fill=black] (-.25,0)--(-.25,2)--(2,2)--(2,0)--cycle;
	\draw[fill=black] (1,-.5)--(6,-.5)--(6,1)--(1,1)--cycle;
	\draw[color=white, fill=white] (.25,.5) -- (1.5,.5)--(1.5,1.5)--(.25,1.5)--cycle;
	\draw[fill=black] (.75,1.1) circle (.25);
	\draw[fill=black] (.5,.5)--(.5,.8)--(1,.8)--(1,.5)--cycle;
	\draw[fill=black] (1,-.25) circle (1); %large wheel 
	\draw[fill=black] (2.75,-.75) circle (.5); %small wheel 1
	\draw[fill=black] (2.5,-1)--(6.5,-1)--(6.5,-.9)--(2.5,-.9)--cycle; %line connecting wheels
	\draw[fill=black] (4.5,-.75) circle (.5); %small wheel 2
	\draw[fill=black] (6,-.5) -- (7,-.5)--(6,1)--cycle; %front triangles
	\draw[fill=black] (6.5,-.5) -- (7.3,-.5)--(6.5,.25)--cycle;
	\draw[fill=black] (6.25,-.75) circle (.5); %small wheel 3
	\draw[fill=black] (5.4,1)--(5.8,1)--(5.8,1.5)--(5.4,1.5)--cycle; %smokestack
	\draw[fill=black] (5.4,1.5)--(5.8,1.5)--(6,1.7)--(5.2,1.7)--cycle;
        \draw[fill=white] (2,0)--(6.2,0)--(6,.3)--(2,.3)--cycle; %white accent
	\draw node at (5.1,2.5) {\cloud{1}};
	\draw node at (3.9,3) {\cloud{1}};
	\draw node at (2.25,2.9) {\cloud{1}};
	\end{tikzpicture}%
	}
	
\newcommand{\traincar}[1]{%
\begin{tikzpicture}[scale=#1]
	\draw[fill=black] (1,-.5)--(4.5,-.5)--(4.5,1.5)--(1,1.5)--cycle;
	\draw[fill=white] (1.5,.5)--(2,.5)--(2,1)--(1.5,1)--cycle;
	\draw[fill=white] (2.5,.5)--(3,.5)--(3,1)--(2.5,1)--cycle;
	\draw[fill=white] (3.5,.5)--(4,.5)--(4,1)--(3.5,1)--cycle;
	\draw[fill=black] (1.75,-.75) circle (.5);
	\draw[fill=black] (3.75,-.75) circle (.5);
	\draw[fill=black] (1.75,-1)--(3.75,-1)--(3.75,-.9)--(1.75,-.9)--cycle;
        \draw[fill=black] (1.5,1.5)--(4,1.5)--(4,1.7)--(1.5,1.7)--cycle;
\end{tikzpicture}%
}


%opening
\title{\vspace*{-1.5in}\textbf{Math 2510: Confidence Intervals Flowchart}}
\date{}
%\author{}


\begin{document}
\maketitle
\pagenumbering{gobble}
%\underline{Math 1300-400: Quiz 1}
%\lhead{}
%\chead{April 29, 2015}
%\rhead{Quiz 11}
\begin{center}
\begin{tikzpicture}
    \matrix (m) [matrix of math nodes, row sep=1cm,column sep=.2cm] {
   \node[align=left,anchor=center]{Unknown\\ Parameter:};& & \node[draw,thick,rectangle]{\mu};& &  \node[rectangle][draw,thick,rectangle]{p}; &  &\node[rectangle][draw,thick,rectangle]{{\mu_1 - \mu_2}^*}; &&  \node[draw,thick,rectangle]{{p_1-p_2}^*};\\
    &\node{\sigma\ \text{known}}; && \node{\sigma\ \text{unknown}}; & \node{\phantom{asdffds}}; &  \node{\sigma_1,\sigma_2\ \text{known}}; && \node{\sigma_1,\sigma_2\ \text{unknown}}; &\node{\phantom{asdfsadffds}};  \\
\node[align=left,anchor=center]{{Calculator}\\ {Function:}}; &\node{\verb|ZInterval|};&&\node{\verb|TInterval|};&\node{\verb|1-PropZInt|};&\node{\verb|2-SampZInt|};&&\node{\verb|2-SampTInt|};&\node{\verb|2-PropZInt|};\\
\node[align=left]{Estimate:};&\node{\overline{x}}; &&\node{\overline{x}};&\node{\hat{p}};& \node{\overline{x}_1 - \overline{x}_2};&&\node{\overline{x}_1 - \overline{x}_2};&\node{\hat{p}_1 - \hat{p}_2};\\
\node[align=left]{Error (E):};&\node{ z_c \frac{\sigma}{\sqrt{n}}};&&\node{t_c \frac{S}{\sqrt{n}}}; & \node{ z_c \sqrt{\frac{\hat{p}\hat{q}}{n}}}; &\node{z_c \sqrt{\frac{\sigma_1^2}{n_1} + \frac{\sigma_2^2}{n_2}}};&&\node{t_c \sqrt{\frac{S_1^2}{n_1} + \frac{S_2^2}{n_2}}};&\node{z_c \sqrt{\frac{\hat{p}_1\hat{q}_1}{n_1} + \frac{\hat{p}_2\hat{q}_2}{n_2}}};\\[-1cm]
&&&\node[align=left,anchor=south]{\small (d.f. = $n-1$)};&&&& \node[align=left,anchor=center]{\small (d.f. = smaller of\\\small $n_1-1$ and $n_2 -1$)};\\
};
\path[->,font=\scriptsize](m-1-3) edge (m-2-2) edge (m-2-4);
\path[->,font=\scriptsize](m-1-7) edge (m-2-6) edge (m-2-8);
\path[->,font=\scriptsize](m-2-2) edge (m-3-2)
     (m-2-4) edge (m-3-4)
     (m-1-5) edge (m-3-5)
     (m-2-6) edge (m-3-6)
     (m-2-8) edge (m-3-8)
     (m-1-9) edge (m-3-9);
\foreach \i in {2,4,5,6,8,9}{
\path[->,font=\scriptsize](m-3-\i) edge (m-4-\i);
\path[->,font=\scriptsize](m-4-\i) edge (m-5-\i);
}
%\path(m-5-4) edge (m-6-4);
\end{tikzpicture}
\end{center}
\vskip2cm
${}^*$Note: These methods should only be used when the two samples are independent, which is the case in Chapter 7.


%\begin{tikzpicture}
%  \matrix (m) [matrix of math nodes,row sep=3em,column sep=4em,minimum width=2em]
%  {
%     F_t(x) & F(x)\\
%     A_t & A \\};
%  \path[-stealth]
%    (m-1-1) edge node [left] {$\mathcal{B}_X$} (m-2-1)
%            edge [double] node [below] {$\mathcal{B}_t$} (m-1-2)
%    (m-2-1.east|-m-2-2) edge node [below] {$\mathcal{B}_T$}
%            node [above] {$\exists$} (m-2-2)
%    (m-1-2) edge node [right] {$\mathcal{B}_T$} (m-2-2)
%            edge [dashed,-] (m-2-1);
%\end{tikzpicture}



%\begin{flushright}
%\train{.1}{10}
%\end{flushright}

\end{document}

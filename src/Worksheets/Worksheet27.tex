\documentclass{article}
\usepackage{graphicx, color, multicol}
\usepackage{hyperref}


\setlength{\textwidth}{6.5in}
\setlength{\textheight}{8.5in}
\setlength{\oddsidemargin}{0in}
\setlength{\evensidemargin}{0in}
\setlength{\parskip}{2ex}
\setlength{\parindent}{0in}

\newcommand{\calcfunc}[1]{\mbox{\texttt{#1}}}

\begin{document}
%To display answers, replace "white" with "red".
\newcommand{\answer}[1]{{\color{red}#1}}

\pagestyle{myheadings}\markright{
CU Boulder \hspace{0.5in} MATH 2510 - Introduction to Statistics }

\begin{center}
\textbf{\underbar{In-class Worksheet 27 - Mini-Final Solutions}}
\end{center}

\begin{enumerate}

\item \answer{C} To find $P(X\geq 6)$, we use the compliment rule and note the compliment to $X\geq 6$ is the event $X\leq 5$. Thus
$$P(X\geq 6) = 1-P(X\leq 5) = 1-\calcfunc{binomcdf}(10,0.7,5) \approx 0.849732.$$

\item \answer{A} Since we fail to reject $H_0$, (which occurs when the P value is greater than $\alpha$), we have insufficient evidence to support the claim made in $H_1$. Thus we should conclude in this example \textbf{there is insufficient evidence to support the claim $\mu > 54.4$}.

\item \answer{A} We are asked to compute a confidence interval for a population mean and we have do not have knowledge of $\sigma$, the population standard deviation. This means we must either assume we are sampling from a normal distribution \textbf{OR} have a sample size of 30 or more to use a $t$-based interval.

After inputting the sample data into $\calcfunc{L}_1$, we can use \texttt{TInterval} with \texttt{Inpt} set to \texttt{Data} we have our confidence interval of $(3.3882, 6.0118)$.

\item \answer{D} We are asked to find a necessary sample size for a confidence interval about a population mean. This means we should use the formula $$n\geq \left( \frac{z_c \cdot \sigma}{E} \right)^2.$$

Using $\calcfunc{invNorm}(0.95,0,1)$ to find the critical $z$-value, we see $$z_c \approx 1.644853626.$$

Combining this with the information given we have $$n \geq \left(\frac{1.644853626\cdot 27}{2.8}\right)^2 \approx 251.57413.$$

Since $n$ must be an integer and greater than $251.57413$, $n$ must be 252 or larger. Note rounding $z_c$ to an insufficient decimal place may give an incorrect final result.

\item \answer{A} A 94\% confidence interval corresponds to a left tail area of $0.94 + \frac{1}{2}\left(1-.94\right)=0.97$ for $z_c$. Thus
$$z_c = \calcfunc{invNorm}\left(0.97,0,1\right)\approx 1.88079.$$

\item \answer{C} By the Central Limit Theorem, the standard deviation of sample means from samples of size 50 is $22/\sqrt{50}$. Thus $$P(X\leq 200) = \calcfunc{normalcdf}\left(-1\calcfunc{E}99, 200, 196, 22/\sqrt{50}\right) \approx 0.900717.$$

\item \answer{B} We are asked to find a conditional probability. The known condition is ``manager is rated as fair" and there are 87 such managers. This will be the denominator of the probability we are asked to compute. The number of \textbf{fair} managers with \textbf{no college background} is 1. This is the numerator of the probability we are asked to compute.  Thus
$$P\left(\mbox{no college background} | \mbox{fair rating} \right) = \frac{1}{87}.$$
An alternative method is to use the formula given in the book for conditional probabilities.

\item \answer{C} A $z$-score corresponds to the number of standard deviations a score is away from the mean. A positive $z$-score is indicative of a score larger than the mean. Thus the exam score is 
$$70 + 2.33\cdot 5 = 81.65.$$

\item \answer{B} We are given a 95\% confidence interval for $\beta$, the slope of the population least-squares line. In particular, we note the lower bound is positive which means as $x$ increases, $y$ should also be increasing. So if $x$ increases by 1 unit, then $y$ will increase by some amount dictated by the confidence interval.

To be precise, we can state: \textbf{We are 95\% confident that if sales revenue increases by \$1 million, then bank charges will increase by an amount between \$15 and \$25}.

\item \answer{A} Since 5 more students scored below average, the class mean will clearly decrease. The fact that standard deviation will increase is less clear. Because all of these new scores are more than 2 standard deviations away from the mean, this will certainly increase the standard deviation.

For those interested, the precise criterion and explanation can be found \href{http://stats.stackexchange.com/questions/73498/value-that-increases-the-standard-deviation}{\color{blue} \textbf{here}}. It is not necessary to know this and this will not be explicitly tested.

\item \answer{D} Since this is a standard normal curve, we know $\mu=0$ and $\sigma =1$. Thus the area is 
$$\calcfunc{normalcdf}(-1.33, 1\calcfunc{E}99, 0, 1) \approx 0.9082408.$$

\item \answer{B} Since we are testing for fitting a uniform distribution, we need to compute the expected column for the $\chi^2$ goodness-of-fit test. The total number of obersvations is 240 and there are 6 possible starting positions. Hence we should expect to see $240\cdot\frac{1}{6}=40$ for the number of wins in each starting position. Putting the observed frequencies into $\calcfunc{L}_1$, expected into $\calcfunc{L}_2$, and using $6-1=5$ degrees of freedom, we see that $\chi^2\calcfunc{GOF-Test}$ gives $\chi^2=6.75$.

\item \answer{D} The strength of linear correlation, measured by $\rho$ or $r$ for populations and samples respectively, is visualized by how tightly packed the dots would be around the least-squares line. Plot D is most suggestive of a strong correlation.

\item \answer{C} Despite the fact the measurements are numbers, their \textbf{natural ordering does not convey ``more or less of a soccer player".} This makes the level of measurement to be nominal.

\item We are comparing two population means through independent sampling, this is clearly unpaired data. As we are ignorant of the population standard deviations, we should use a \texttt{2-SampTTest}. If we call the mean first place price for women $\mu_1$ and $\mu_2$ for men, $H_0$ states $\mu_1=\mu_2$ and $H_1$ states $\mu_1 < \mu_2$. 

Since our P-value is less than $\alpha$, we reject $H_0$. The answer key explains the rest.

\item From the graph, we see the midpoint of each interval is $\{55,60,65,\ldots, 80, 85\}$. The answer key gives a reasonable frequency reading for each of these interval. Putting the midpoints into $\calcfunc{L}_1$ and frequencies into $\calcfunc{L}_2$, use \texttt{1-Var Stats} to display the mean, median and mode.

\item Treating $x$ as the temperature and $y$ as consumption, we put the $x$-values into $\calcfunc{L}_1$ and $y$-values into $\calcfunc{L}_2$. Use \texttt{LinRegTTest} will compute both the P-value and the least-squares line of best fit.

It is important to distinguish which set of data values are ``x"-vales and which set are ``y"-values. On the final exam, this will be made clear and be sure to \textbf{pay attention which one is which}.

\item Putting the group data into $\calcfunc{L}_1$,$\calcfunc{L}_2$, and $\calcfunc{L}_3$ and using $$\calcfunc{ANOVA}\left(\calcfunc{L}_1,\calcfunc{L}_2,\calcfunc{L}_3\right)$$
will give the necessary information for the solution found in the answer key.

Again, since the P-value is less than the level of significance, we reject $H_0$ and have sufficient evidence to support $H_1$.

\end{enumerate}

\end{document}
\documentclass{article}
\usepackage{color}

\setlength{\textwidth}{6.5in}
\setlength{\textheight}{8.0in}
\setlength{\oddsidemargin}{0in}
\setlength{\evensidemargin}{0in}
\setlength{\parskip}{2ex}
\setlength{\parindent}{0in}

%To display answers, replace "white" with "red" here;
\newcommand{\answer}[1]{\color{red}#1}

\begin{document}
\pagestyle{myheadings}\markright{
CU Boulder \hspace{0.5in} MATH 2510 - Introduction to Statistics }

\begin{center}

\textbf{\underbar{In-class Worksheet 1}}
\end{center}

\begin{enumerate}
\item You are conducting a study of students doing work-study jobs on your campus.  Among the questions on the survey instrument are:
	\begin{itemize}
	\item How many hours are you schedule to work each week?  (Answer to the nearest hour.)
	\item How applicable is this work experience to your future employment goals?  (Respond using the following scale: 1 = not at all, 2 = somewhat, 3 = very)
	\end{itemize}
	
	\begin{enumerate}
	
	\item Suppose you take random samples from the following groups: freshmen, sophomores, juniors, seniors, and others.  What kind of sample technique are you using?  Explain. 
	
	{\answer Stratified sampling is being used because the population is divided into distinct subgroups (strata) from which random samples are then drawn.}
	\vfill
	
	\item Describe the individuals of this study. 
	
	{\answer The individuals are the students on campus with work-study jobs.}
	\vfill
	
	\item What is the variable for the first question on the survey?  Classify the variable as quantitative or qualitative.  What is the level of the measurement?  Explain. 
	
	{\answer The variable is the number of hours scheduled.  This is a quantitative variable with a ratio level of measurement, because it makes sense to say that one student works ``3 times" more hours than another.}
	\vfill
	
	\item What is the variable for the second question on the survey?  Classify the variable as quantitative or qualitative.  What is the level of the measurement?  Explain. 
	
	{\answer The variable is the rating of applicability.  This is a qualitative variable with an ordinal level of measurement, because the is a natural ordering of ``least" to ``most".}
	\vfill
	
	\item Is the proportion of responses ``3 = very" to the second question a statistic or a parameter?  Explain. 
	
	{\answer This proportion is a statistic, because it describing just a sample, rather than the entire population.}
	\vfill
	
	\item Would it be appropriate to generalize the results of your study to all work-study students in the nation?  Explain. 
	
	{\answer Because the sample frame is restricted to just one campus, it would not be appropriate to generalize these results to all work-study students in the nation.}
	\vspace{1cm}
	
	\end{enumerate}
		
\newpage


\item Suppose that you are conducting a study to compare firefly populations exposed to normal daylight/darkness conditions with firefly populations exposed to continuous light (24 hours of day).  You set up two firefly colonies in a laboratory environment.  The two colonies are identical except that Colony 1 is exposed to normal daylight/darkness conditions and Colony 2 is exposed to continuous light.  Each colony is populated with the same number of mature fireflies.  After 72 hours, you count the number of living fireflies in each colony. 

	\begin{enumerate}
	
	\item Is this an experiment or an observational study?  Explain. 
	
	{\answer This is an experiment, because a treatment (continuous lighting) is being imposed on one colony.}
	\vfill
	
	\item Is there a control group?  Is there a treatment group?  If so, which colony is which? 
	
	{\answer The control group is the colony receiving normal daylight/darkness conditions.  The treatment group is the colony receiving light 24 hours per day.}
	\vfill
	
	\item What is the variable in this study? 
	
	{\answer The variable is the number of fireflies living at the end of 72 hours. }
	\vfill
	
	\item What is the level of measurement of this variable? 
	
	{\answer This variable has a ratio level of measurement.}
	\vfill
	
	\end{enumerate}

\item A tooth-whitening gel is being tested for effectiveness.  A group of 85 adults have volunteered to participate.  Of these 43 are randomly chosen to receive the gel with the tooth-whitening chemicals.  The other 42 are given a similar looking gel with none of the chemicals. 

	\begin{enumerate}
	
	\item Describe the control group and the treatment group. 
	
	{\answer The control group is the group of 42 volunteers receiving the gel with no tooth-whitening chemicals.  The treatment group is the group of 43 volunteers receiving the gel that does contain the tooth-whitening chemicals.}
	\vfill
	
	\item Is a placebo being used? 
	
	{\answer A placebo (the gel without the active chemicals) is being used.}
	\vfill
	
	\item A standard method will be used to evaluate the whiteness of teeth for all participants.  Why might a double-blind design be used in this case? 
	
	{\answer In the case of a double-blind study, neither the volunteer nor the dental professional would be aware of which gel contained the tooth-whitening chemical and which did not.  By concealing this information, it would reduce the likelihood of biased volunteer participation (not applying the gel as prescribed because they know it contains no whitening chemicals), as well as potential bias from the evaluator if they knew the gel used contained the chemical or not.}
	\vfill
	
	\end{enumerate}

	
\end{enumerate}

\vfill

\end{document}

